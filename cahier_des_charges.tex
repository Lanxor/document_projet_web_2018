\section{Présentation}

  \begin{paragraphe}
  	Le projet que nous réalisons est un site de critique de musique.\par
    Il met à disposition un large catalogue comportant de nombreux artistes
    pour lesquels on retrouve une brève description ainsi que leur discographie.
    Pour chaque album on donne les chansons présentes et quelques informations
    supplémentaires (durée, date de parution, artistes, genre, etc). \par
    Il est également possible de s’identifier afin de pouvoir rédiger des
    commentaires et des évaluations pour chaque album ou titres présent sur le site.
  \end{paragraphe}

  \begin{paragraphe}
    Ainsi, le site final est un recueil d’albums pour lesquels on retrouve
    quelques informations et de nombreux avis d’utilisateurs.
  \end{paragraphe}

  \subsection{Systèmes comparables}

  \begin{paragraphe}
    Il existe le site Pitchfork (https://pitchfork.com/), site anglo-saxon
    spécialisé dans le revue d’albums de tout genre musical. Cependant ce site
    offre des revues rédigées par des auteurs spécialisés appartenant à la
    société Pitchfork et n’offrent pas plus de données sur l’album mis à part
    la revue.\par
    Le site Rotten Tomatoes (https://www.rottentomatoes.com/), site de revue de
    films, se rapproche plus de notre projet car il permet aux utilisateurs de
    donner un avis et fournit donc la note moyenne accordée par les
    utilisateurs à une œuvre, et donne également des données sur le film
    (durée, date de parution).
  \end{paragraphe}

  \subsection{Profils et rôles utilisateurs}

  \begin{paragraphe}
    Nous utilisons trois profils d’acteurs à travers ce projet: le profil
    anonyme, le profil administrateur et le profil utilisateur.\par
    En tant que créateurs du site, nous sommes les administrateurs de celui-ci
    et avons donc la charge de mettre le site en place, de le rendre
    fonctionnel et de remplir le catalogue d'artistes (saisir les artistes, les
    albums, les pistes ainsi que les informations pour chaque).\par
    Une fois le site fonctionnel et rempli, tous les profils sur le site pourront consulter les pages.
    Les profils utilisateurs (à qui on demande une identification et une authentification) pourront eux rédiger des évaluations et accéder à des paramètres relatifs à leur compte.
  \end{paragraphe}

  \subsection{Contraintes}

  \begin{paragraphe}
    Les difficultés de ce projet sont le temps disponible ainsi que le volume
    de la base de données à fournir avant de pouvoir déclarer le site comme
    fonctionnel. De plus, il faut effectuer une phase de conception approfondie et la plus complète possible afin de ne pas avoir de soucis lors de l'implémentation.
  \end{paragraphe}

\section{Fonctionnalités}

  \subsection{Terminologie}

    \begin{paragraphe}
      Nous allons tout d'abord spécifier quelques termes utilisés dans ce cahier des
      charges.
      \begin{itemize}
        \item \textbf{Artiste :} Définit le compositeur de morceaux de musiques, en solo ou au sein d'un groupe.
        \item \textbf{Album :} Est un recueil de morceaux de musique d'un ou
          plusieurs artistes.
        \item \textbf{Morceau de musique :} (Ou titre) Est une musique faisant partie d'un
          album ou d'un single.
        \item \textbf{Single :} Est une musique ne faisant pas partie d'un album
          mais directement relié à un artiste.
        \item \textbf{Groupe :} Regroupement d'au moins deux artistes ayant composé ensemble des morceaux de musique.
        \item \textbf{Genre :} Caractéristique pouvant être associée à un morceau de musique.
        \item \textbf{Récompense :} Récompense décernée à un artiste ou groupe.
        \item \textbf{Métadonnées :} Ensemble d'éléments caractérisant un artiste, un groupe, un album, une récompense ou un morceau de musique.
        \begin{itemize}
            \item \textbf{Artiste :} Nom, Prénom, Nom de scène, Date de naissance, description et url d'une photo.
            \item \textbf{Groupe :} Nom du groupe, Date de formation, description et url d'une photo.
            \item \textbf{Album :} Nom de l'album, Date de sortie, description et url de la pochette.
            \item \textbf{Titre :} Titre, durée, date de sortie et description.
            \item \textbf{Recompense :} Intitulé, date de remise et description.
        \end{itemize}
      \end{itemize}
    \end{paragraphe}

  \subsection{Parties de l'application}

    \begin{paragraphe}
      Il existe deux parties dans l'application. Une partie publique accessible à tout utilisateur du site et une partie
      privée réservée aux utilisateurs administrateurs.
    \end{paragraphe}

    \begin{paragraphe}
      \textbf{La partie public} est la vitrine de l'application. C'est ici que l'on
      va retrouver les différentes fonctionnalités citées plus bas. On peut
      ainsi rechercher et consulter les pages artistes, albums, groupes ou titres.
      Dans cette partie, on retrouvera deux sous parties qui seront selon le profil de l'utilisateur.
      Si ce dernier est identifié et connecté il aura également accès au système d'évaluation.

      \textbf{La partie privée} est réservée au modérateur de l'application web, c'est-à-dire aux profils administrateurs.
      Elle permet entre autre de gérer toute la base de données des artistes,
      album, groupes, single et métadonnées de ces derniers. On peut ajouter, modifier
      ou supprimer une entrée dans la base de données.
      Elle offre également un accès de gestion des comptes.
    \end{paragraphe}

  \subsection{Description des fonctionnalités}

    \begin{paragraphe}
      	Le site offre la possibilité de créer un compte, de consulter des pages
        (artistes, albums), de rédiger des évaluations et des revues.
    \end{paragraphe}

    \begin{paragraphe}
      Voici les différentes fonctionnalités disponible dans l'application web.
    \end{paragraphe}

    \begin{paragraphe}
      Les fonctionnalités suivantes sont disponibles pour n'importe quel utilisateur du site.
    \end{paragraphe}

    \begin{paragraphe}
      \begin{itemize}
        \item Effectuer une recherche d'un artiste, d'un groupe, d'un album ou d'un single
          par son nom.
        \item Consulter la description d'un artiste. On retrouve son prénom, son nom, sa
          date de naissance, une brève description et sa discographie. Si l'on
          recherche un groupe, sa description sera adaptée et l'on retrouvera,
          en plus des informations classiques, les différents artistes qui
          composent le groupe.
        \item Consulter la description d'un album. On retrouve son titre,
          l'artiste ou groupe qui l'a composé, son année de parution, la liste des morceaux de musique
          qu'il compte ainsi qu'une description et une photo de sa couverture.
        \item Consulter les métadonnées d'un morceau de musique. On retrouve
          le nom, l'album et l'artiste ou groupe rattaché, sa durée, l'année de composition
          et son genre s'il est renseigné.
        \item Créer un compte utilisateur.
        \item Se connecter à un compte enregistré.
        \item Modifier son mot de passe.
        \item Supprimer son compte utilisateur.
      \end{itemize}
    \end{paragraphe}

    \begin{paragraphe}
      Pour les utilisateurs identifiés et connectés, les fonctionnalités suivantes sont également disponibles.
    \end{paragraphe}

    \begin{paragraphe}
      \begin{itemize}
        \item Déposer une critique sur la page d'un album ou
          d'une musique sous la forme d'une note entière sur 5 pouvant être accompagnée d'un commentaire.
        \item Supprimer une critique dont on est l'auteur.
      \end{itemize}
    \end{paragraphe}

    \begin{paragraphe}
      Pour la partie privée voici les fonctionnalités de gestion de la base de
      données.
    \end{paragraphe}

    \begin{paragraphe}
      \begin{itemize}
        \item Créer un artiste avec toutes ses métadonnées.
        \item Créer un album avec ses titres et ses métadonnées.
        \item Créer un morceau de musique avec toutes ses métadonnées.
        \item Créer un groupe avec ses artistes et toutes ses métadonnées.
        \item Créer une récompense à un artiste avec toutes ses métadonnées.
        \item Modifier les métadonnées d'un artiste.
        \item Modifier les métadonnées d'un album.
        \item Modifier les métadonnées d'un morceau de musique.
        \item Modifier les métadonnées d'une récompense.
        \item Modifier les métadonnées d'un groupe.
        \item Supprimer un artiste (supprime aussi ses albums et les titres de
          musique qu'il a composé).
        \item Supprimer un album d'un artiste (supprime aussi les titres qui le composent).
        \item Supprimer un morceau de musique d'un album.
        \item Supprimer une récompense d'un artiste.
        \item Supprimer un groupe (ne supprime pas les artistes qui le composent).
        \item Supprimer un compte utilisateur.
        \item Créer un compte utilisateur.
        \item Modifier les mots de passe d'autres comptes utilisateurs.
        \item Accorder ou retirer les droits d'administration à un compte utilisateur classique.
        On notera que les comptes administrateurs ne peuvent pas se retirer les droits d'administration à eux-même ni se supprimer mutuellement depuis le panneau prévu à cet effet.
      \end{itemize}
    \end{paragraphe}

\section{Scénarios}

  \begin{paragraphe}
    Voici différents scénarios possibles d'un utilisateur interagissant avec
    l'application web.
  \end{paragraphe}

  \begin{paragraphe}
    \textbf{Scénario 1 :}
    Un utilisateur cherche le nom d'une musique pour obtenir des informations
    sur ce titre, comme l'album auquel il appartient, des informations sur les
    auteurs de ce morceau, la date de parution, etc. Il regarde également les évaluations
    des internautes dans la section commentaire pour se faire un avis général.
  \end{paragraphe}

  \begin{paragraphe}
    \textbf{Scénario 2 :}
    Pour l'anniversaire de sa fille Sophie, Teresa veut lui offrir un album de
    musique. Teresa n'est cependant pas très familière avec les artistes
    d'aujourd'hui. Elle va donc consulter l'application en cherchant des
    artistes plus populaires du moment en se basant sur les évaluations ainsi que les dates de parution d'album. Elle trouve ainsi l'artiste
    M. Pokora qui a sorti un album récemment. Elle est donc convaincue et ira
    acheter l'album pour sa fille.
  \end{paragraphe}

  \begin{paragraphe}
    \textbf{Scénario 3 :}
    Ludovic est un fan de Ed sheeran et souhaite mettre un commentaire sur le
    site.
    Il met un commentaire sur la chanson "Casltle on the Hill" qui fait partie
    du 3ème album de l'artiste.
  \end{paragraphe}

  \begin{paragraphe}
    \textbf{Scénario 4 :}
    Un utilisateur possédant un compte et s'étant connecté cherche un album qu'il
    a écouté et note chacun des morceaux en rédigeant ou non un commentaire.
  \end{paragraphe}

  \begin{paragraphe}
    \textbf{Scénario 5 :}
    Une personne se connecte sur la page d’accueil, effectue une recherche pour
    trouver la page de son groupe favori. Sur cette page il découvre la liste
    des albums parus par ce groupe, et décide de rédiger une évaluation de cette
    œuvre. Il s’identifie, rédige un avis et quitte la page. Sa session se
    ferme ensuite par le serveur après avoir remarqué une longue absence.
  \end{paragraphe}

  \begin{paragraphe}
    \textbf{Scénario 6 :}
    Une personne veut rédiger un avis sur un album, trouve la page
    correspondante mais ne possède pas de compte. Il décide donc de se créer un
    compte afin de pouvoir rédiger son évaluation.
  \end{paragraphe}
